\documentclass{article}
% GENERAL
\usepackage{setspace,mathtools,amsfonts,amsmath,amsthm,amssymb,hyperref}
\usepackage{tikz,epigraph,pgfplots}  % For trees
\usepackage[utf8]{inputenc}
\usepackage{tikz-qtree,tikz-qtree-compat}
\usepackage{forest}

% FOR SOURCE CODE
\usepackage{listings}
% Default settings for code listings
\lstset{frame=tb,
  aboveskip=3mm,
  belowskip=3mm,
  showstringspaces=false,
  columns=flexible,
  basicstyle={\small\ttfamily},
  numbers=none,
  numberstyle=\tiny\color{gray},
  keywordstyle=\color{blue},
  commentstyle=\color{dkgreen},
  stringstyle=\color{mauve},
  frame=single,
  breaklines=true,
  breakatwhitespace=true
  tabsize=4
}

% FOR TREE DRAWING
\pgfplotsset{compat=newest}
\usetikzlibrary{shapes.geometric,arrows,fit,matrix,positioning}
\tikzset
{
      treenode/.style = {circle, draw=black, align=center, minimum size=1cm}
}

% MARGINS
\usepackage[left=1in,top=1in,right=1in,bottom=1in]{geometry}
\onehalfspacing

\begin{document}
\title{CMPT 419 Assignment 2}
\author{Colin Woodbury\\ 301238755\\ cwoodbur@sfu.ca}
\date{\today}
\maketitle

% --- TABLE OF CONTENTS ---
\tableofcontents
\clearpage
% -------------------------

\section{Linear Model for Classification}

\begin{center}
  \includegraphics[scale=0.75]{dbs}
\end{center}

A point is assigned to class $C_k$ if $y_k(\mathbf{x}) > y_j(\mathbf{x})$
for all $j \neq k$. Since we already know the decision boundaries and
points that lay on them, we need only to choose three planes of the
form $y_k(\mathbf{x}) = \mathbf{w}^T_k\mathbf{x} + w_k0$ that match the
above criteria.\\

The points $(-1,0,0)$ and $(0,-1,0)$ lie on the decision boundary of
Class 1 and 2. Since there are an infinite number of planes that intersect
this boundary, we choose an arbitrary third point to constrain the solution,
such that this plane will yield the highest height among the three, given
$\mathbf{x}$ values in the Class 1 area.
Let this point be $(-1,-1,1)$. Solving for the equation of a plane,
we find:

\begin{align*}
  y_1(\mathbf{x}) &= -1x_1 - 1x_2 - 1\\
  \text{Given: } \mathbf{w^T} &= (-1,-1)\\
  w_0 &= -1
\end{align*}

We solve for $y_3$ in a similar fashion. The two points on this plane
are $(1,0,0)$ and $(0,1,0)$. Let a third points be $(1,1,1)$. Solving
for the plane in the same way, we have:

\begin{align*}
  y_3(\mathbf{x}) &= 1x_1 + 1x_2 - 1\\
  \text{Given: } \mathbf{w^T} &= (1,1)\\
  w_0 &= -1
\end{align*}

$y_2$ is the simplest. Since both $y_1$ and $y_3$ have negative values
in $\mathbf{w}^T\mathbf{x} + w_0$ space within the Class 2 area,
the plane for Class 2 can sit parallel to the feature space. Specifically:

\begin{align*}
  y_2(\mathbf{x}) &= 0
\end{align*}

\section{Kernels}

\begin{enumerate}
\item hi
\item hi
\end{enumerate}

\section{Logistic Regression}

\begin{enumerate}
\item Plots
  \begin{center}
  \includegraphics[scale=0.75]{3-1b}
  \includegraphics[scale=0.75]{3-1c}
\end{center}

  Why do these oscillate?
  
\item Plot
  \begin{center}
    \includegraphics[scale=0.75]{3-2}
  \end{center}
\item hi
\item hi
\end{enumerate}

\section{Kernelized Perceptron}



\end{document}
