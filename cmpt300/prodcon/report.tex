\documentclass[titlepage]{article}

% GENERAL
\usepackage[usenames,dvipsnames]{color}
\usepackage{setspace,graphicx,fancyhdr,hyperref,amsmath,tikz,epigraph}
\usepackage{listings,enumitem}
\usepackage{framed}
\usepackage[utf8]{inputenc}

% MARGINS
\usepackage[left=1in,top=1in,right=1in,bottom=1in]{geometry}
\onehalfspacing

% LANGUAGE
\lstdefinelanguage{c}{
  morekeywords={for,while,NULL,struct},
  sensitive=true,
  morecomment=[l]{//},
  morestring=[b]"
}

\definecolor{dkgreen}{rgb}{0,0.6,0}
\definecolor{gray}{rgb}{0.5,0.5,0.5}
\definecolor{mauve}{rgb}{0.58,0,0.82}

% Default settings for code listings
\lstset{frame=tb,
  language=c,
  aboveskip=3mm,
  belowskip=3mm,
  showstringspaces=false,
  columns=flexible,
  basicstyle={\small\ttfamily},
  numbers=none,
  numberstyle=\tiny\color{gray},
  keywordstyle=\color{blue},
  commentstyle=\color{dkgreen},
  stringstyle=\color{mauve},
  frame=single,
  breaklines=true,
  breakatwhitespace=true
  tabsize=4
}

% DOCUMENT
\begin{document}

\title{CMPT 300 Assignment 2 - The Foundry}
\author{Colin Woodbury\\ 301238755\\ cwoodbur@sfu.ca}
\date{\today}
\maketitle

\section{Design Discussion}

\subsection{Threads}
The Foundry has a main thread, three Generator threads, and a number of
Operator threads, depending on how many the user specified.

\subsubsection{nCurses - Avoiding Race Conditions}
It turns out that certain nCurses functions aren't thread safe.
In particular, functions like:

\begin{lstlisting}
mvprintw(y, x, "FooBar");
\end{lstlisting}

which access some master cursor position, can cause race conditions
if called in multiple threads. Characters will render where they aren't
supposed to and generally globber the animation. The solution to this
is to limit calls like this to the main thread.

\subsection{Mutexes and Avoiding Deadlock}
The Foundry has three main bodies of shared resources, and mutexes to manage
each of them. Below only the mutexes and their thread sleep conditions are
listed. See source code for other shared resources.

\subsubsection{Metals Resources}
\begin{lstlisting}
pthread_mutex_t metal_mutex;
pthread_cond_t metals_full;   // The metals queue is full.
pthread_cond_t metals_empty;  // The metals queue is empty.
\end{lstlisting}

\begin{itemize}
  \item Only one Generator can place a metal in the queue at a time.
  \item Generator threads will sleep with the \emph{metals\_full} condition
    when the metals queue has reached its maximum capacity.
  \item Operators trying to obtain metals will sleep with the
    \emph{metals\_empty} condition when there aren't any available.
\end{itemize}

\subsubsection{Tools Resources}
\begin{lstlisting}
pthread_mutex_t tools_mutex;
pthread_cond_t tools_gone;  // Who took all the damn tools?
\end{lstlisting}

Here, we use a shared \emph{tool\_counter} to indicate how many tools
are still available (instead of using some \emph{Tool} struct and passing
it to the Operator).
Paired with the mutex above this makes for a good pseudo-semaphore.

\begin{itemize}
  \item Only one Operator can attempt to take a tool at a time.
  \item Operators can only take one tool at a time.
  \item If an Operator has no tools and wants one, he will wait with
    the \emph{tools\_gone} condition.
  \item \textbf{To avoid Dining Philosophers}: If an Operator has one tool
    and attempts to take another, but finds none, he will give his
    first tool up.
\end{itemize}

\subsubsection{Alloys Resources}
\begin{lstlisting}
pthread_mutex_t alloy_mutex;
\end{lstlisting}

\begin{itemize}
  \item Only one Operator can put an Alloy in the output queue at a time.
  \item If the Alloy an Operator would make is the same as the most recent
    one made, he will give up and put back the resources he took.
  \item If the Alloy an Operator would make would throw off the
    ``10-difference'' constraint, he will give up and put back the resources
    he took.
\end{itemize}

\subsubsection{Other Resources}
\begin{lstlisting}
bool paused = false;
struct timespec t;  // Controls overall speed of the system.
\end{lstlisting}

These values are global and are read by every thread to control the
execution speed of the simulation. 
They are only written
to by the main thread, although there is no mutex protecting them (bad?).

\end{document}
