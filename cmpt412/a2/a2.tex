\documentclass{article}
% GENERAL
\usepackage{setspace,mathtools,amsfonts,amsmath,amsthm,amssymb,hyperref}
%\usepackage{tikz,epigraph,pgfplots}  % For trees
\usepackage[utf8]{inputenc}
\usepackage{tikz-qtree,tikz-qtree-compat}
\usepackage{forest}

% FOR SOURCE CODE
\usepackage{listings}
\definecolor{mauve}{rgb}{0.58,0,0.82}
\definecolor{dkgreen}{rgb}{0,50,0}
\lstset{frame=tb,
  aboveskip=3mm,
  belowskip=3mm,
  showstringspaces=false,
  columns=flexible,
  basicstyle={\small\ttfamily},
  numbers=none,
  numberstyle=\tiny\color{gray},
  keywordstyle=\color{blue},
  commentstyle=\color{dkgreen},
  stringstyle=\color{mauve},
  frame=single,
  breaklines=true,
  breakatwhitespace=true
  tabsize=4
}

% MARGINS
\usepackage[left=1in,top=1in,right=1in,bottom=1in]{geometry}
\onehalfspacing

\begin{document}
\title{CMPT 412 - Assignment 2 : Photometric Stereo}
\author{Colin Woodbury\\ 301238755\\ cwoodbur@sfu.ca}
\date{\today}
\maketitle

% --- TABLE OF CONTENTS ---
\tableofcontents
\clearpage
% -------------------------

\section{Methods}
As advised directly from Dr. Funt, all images were grayscaled before
further processing. This produced better results than attempting to
handle colour directly.

\subsection{Lookup Table}
The first task was to find the \fbox{center} and \fbox{radius} of the
sphere in the image. As suggested in the assignment specification, I used
\emph{ginput}. With these two pieces of data we are able to determine
if a pixel in the image belongs to the sphere with some simple vector
math. Only if it does do we consider it for addition to the LUT.\\

I implemented solutions with both intensity ratios and without. Using just
two grayscale values for my indices (i.e. E1 and E2) instead of ratios
(E1/E2, E2/E3) produced more accurate results. In this respect, only
two images were necessary instead of three.\\

For missing values in the LUT, we interpolate values with
the \emph{griddata} function. The interpolated F and G spaces look like:

\begin{center}
  \includegraphics{Xs}
  \includegraphics{Ys}
\end{center}

While small at this resolution, the space itself is fairly textured.

\subsection{Normals for new Images}
Fairly straightforward, given the LUT. One thing we need to watch for is
a \emph{threshold} for dark pixels. Simply checking for black is not enough,
as our final quiver plot will be filled with normals approximately equal to
$(0,0,-1)$.

\subsection{Quiver Plot}
Since our images are fairly large, producing a normal for each pixel
is computationally expensive. Instead, we average the normals of each
10-by-10 block of pixels.

\section{Results}

My solution produces the following quiver plots:

\begin{center}
  \includegraphics[scale=0.5]{quiver-sphere}
  \includegraphics[scale=0.5]{quiver-cone}
  \includegraphics[scale=0.5]{quiver-cylinder}
  \includegraphics[scale=0.5]{quiver-ellip}
  \includegraphics[scale=0.5]{quiver-hex1}
  \includegraphics[scale=0.5]{quiver-hex2}
\end{center}

As can be seen, the solution can recover the surface orientations for
all the test images.

\section{Matlab Code}

\subsection{lut.m}

\lstinputlisting[language=Matlab]{lut.m}

\subsection{normals.m}

\lstinputlisting[language=Matlab]{normals.m}

\subsection{quiv.m}

\lstinputlisting[language=Matlab]{quiv.m}

\subsection{demo.m}

\lstinputlisting[language=Matlab]{demo.m}

\end{document}
