\documentclass{article}
% GENERAL
\usepackage{setspace,mathtools,amsfonts,amsmath,amsthm,amssymb,hyperref}
%\usepackage{tikz,epigraph,pgfplots}  % For trees
\usepackage[utf8]{inputenc}
\usepackage{tikz-qtree,tikz-qtree-compat}
\usepackage{forest}

% FOR SOURCE CODE
\usepackage{listings}
\definecolor{mauve}{rgb}{0.58,0,0.82}
\definecolor{dkgreen}{rgb}{0,50,0}
\lstset{frame=tb,
  aboveskip=3mm,
  belowskip=3mm,
  showstringspaces=false,
  columns=flexible,
  basicstyle={\small\ttfamily},
  numbers=none,
  numberstyle=\tiny\color{gray},
  keywordstyle=\color{blue},
  commentstyle=\color{dkgreen},
  stringstyle=\color{mauve},
  frame=single,
  breaklines=true,
  breakatwhitespace=true
  tabsize=4
}

% MARGINS
\usepackage[left=1in,top=1in,right=1in,bottom=1in]{geometry}
\onehalfspacing

\begin{document}
\title{CMPT 412 - Assignment 3 : Colour Based Tracking}
\author{Colin Woodbury\\ 301238755\\ cwoodbur@sfu.ca}
\date{\today}
\maketitle

% --- TABLE OF CONTENTS ---
\tableofcontents
\clearpage
% -------------------------

\section{Methods}
\subsection{YCbCr over RGB}
Experimentally, using YCbCr Histograms showed itself to be more accurate
than using RGB. By virtue of the colour system, the luminance portion is
already separated into its own channel. By discarding that outright,
we are left with a smaller two-axis histogram. This also frees us from
the need for a colour constancy algorithm as is needed with RGB.\\

This is also better for dealing with JPG images, as they are stored in
YCbCr by default. The downside for this particular project is that
the BMP images we are working with are all RGB, and thus require conversion.

\subsection{32-by-32 bins}
Assuming a CbCr histogram, experiments have shown 32x32 bins to be
the most accurate when distinguishing between multiple target images.

As a convention, we ignore any pixels with a Y component value of 20
or less, as these are considered too dark to be useful.

\section{Results}

\subsection{Location via Backprojection}
After a number of tweaks and corrections, simple backprojection performs
quite well for locating known objects within a collage.

\begin{center}
  \includegraphics[scale=0.65]{demo-result}
\end{center}

Out of 9 objects searched for, the algorithm only makes one mistake.
The Care Bears image (mid-bottom) is identified as the mid-top pajamas.

\subsection{Object Tracking via Backprojection}
While slow, this method isn't terrible it terms of accuracy.
Using the following model:

\begin{center}
  \includegraphics[scale=0.5]{cup-model}
\end{center}

it traces the following path through the 50 frames of the video:

\begin{center}
  \includegraphics[scale=0.75]{hist-track-path}
\end{center}

\section{Further Work}
Based on Color Indexing by Swain and Ballard, I have written a Haskell
library \url{https://github.com/fosskers/colour-hist} which implements their
algorithms. Currently only Histogram Intersection is implemented, but I am
happy with the terseness (intersection is one line) and performance of the
code. For Backprojection, I will need to create another library for simple
Matrix convolution, as there doesn't seem to be an implementation using the
ubiquitous Haskell
\emph{Vector} type.

\section{Matlab Code}

\subsection{histogram.m}

\lstinputlisting[language=Matlab]{histogram.m}

\subsection{ratio.m}

\lstinputlisting[language=Matlab]{ratio.m}

\subsection{bin.m}

\lstinputlisting[language=Matlab]{bin.m}

\subsection{backproj.m}

\lstinputlisting[language=Matlab]{backproj.m}

\subsection{demo.m}

\lstinputlisting[language=Matlab]{demo.m}

\subsection{histtrack.m}

\lstinputlisting[language=Matlab]{histtrack.m}

\subsection{trackdemo.m}

\lstinputlisting[language=Matlab]{trackdemo.m}

\end{document}
