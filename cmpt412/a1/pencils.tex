\documentclass{article}
% GENERAL
\usepackage{setspace,mathtools,amsfonts,amsmath,amsthm,amssymb,hyperref}
\usepackage{tikz,epigraph,pgfplots}  % For trees
\usepackage[utf8]{inputenc}
\usepackage{tikz-qtree,tikz-qtree-compat}
\usepackage{forest}

% FOR SOURCE CODE
\usepackage{listings}
\lstset{frame=tb,
  aboveskip=3mm,
  belowskip=3mm,
  showstringspaces=false,
  columns=flexible,
  basicstyle={\small\ttfamily},
  numbers=none,
  numberstyle=\tiny\color{gray},
  keywordstyle=\color{blue},
  commentstyle=\color{dkgreen},
  stringstyle=\color{mauve},
  frame=single,
  breaklines=true,
  breakatwhitespace=true
  tabsize=4
}

% MARGINS
\usepackage[left=1in,top=1in,right=1in,bottom=1in]{geometry}
\onehalfspacing

\begin{document}
\title{CMPT 412 - Assignment 1 : Pencil Detection}
\author{Colin Woodbury\\ 301238755\\ cwoodbur@sfu.ca}
\date{\today}
\maketitle

% --- TABLE OF CONTENTS ---
\tableofcontents
\clearpage
% -------------------------

\section{Journal}
\subsection{2016 January 18}
Random thoughts, having seen the test images:
\begin{itemize}
\item Could the presence of shadows be used to discover "on-top" pencils?
\item I need to think of a way to represent a pencil's location. The centre
  isn't good enough, given that a pencil isn't a circle like a penny.
\item Pencils don't have to be their usual off-yellow. They can also be
  multicoloured.
\item Pencils don't always have the same length.
\item Background could be noisy. Smooth out the image?
\item Probably no point in trying to detect the eraser, or the metal holding it.
\end{itemize}

\subsection{2016 January 21}
I explored the \emph{Image Enhancement} kit. After doing some tests with the
famous \emph{Lena} image, \emph{imgaussfilt} seems to be ideal for smoothing.
Experimentally, $\sigma = 1$ seems to give smoothing that is pleasant to the
eyes, but we'll see whether that or the default of 0.5 is better for actual
edge detection.\\

(Later) I've settled on the default of 0.5.

\subsection{2016 January 25}
I've poked more around the toolkit, and found some nice things to try:
\begin{itemize}
  \item imcontour: Draws a coloured, fairly accurate contouring of the pencil
  \item imhist: Gives a bell-curve of colour values that appear in the image
  \item bwperim: Draws a black and white perimeter around objects in an image.
  \item edge(I,'log',n): Requires large amounts of parameter tweaking to
    perform even marginally well.

\end{itemize}

\subsection{2016 January 2x}
\emph{multithresh} (which I found quite by accident) is the way to go for
removing noise. I should be able to use it to simplify the image, then do
edge detection on that to find the pencils.

\subsection{2016 February 1}
Praise be to \emph{multithresh}! I have everything working now, and
the best part is that I'm not doing any edge detection. My \emph{detect}
function can find most pencils in an image, and will produce a single line
representing said pencil. The carpet image murders my technique, however,
and it also can't tell which pencils are above or below which.

\section{Pencil Detection}

\subsection{Methods}
Our solution has three steps:
\begin{enumerate}
\item Noise reduction
\item Shape detection
\item Line discovery
\end{enumerate}

Note that edge detection is but one way of arriving at the ``detected shapes''
we're looking for. Indeed, in my final solution I employ no direct edge
detection.

\subsubsection{Noise Reduction - Gaussian Filtering}
Image artifacts (noise) make shape detection more difficult. After some
experimention, I settled on the function \emph{imgaussfilter} to help
smooth the image beforehand. While we have a choice of $\sigma$ to affect
the Gaussian, no choice different from the default showed any performance
improvement.

\subsubsection{Shape Detection - Edge Detection}
A quick preruse of the Matlab documentation shows a number of different
algorithms available for edge detection. None that I tried gave me
convincing results, and working with those results would have been
more of a headache than \emph{multithresh} turned out to be.

\subsubsection{Shape Detection - Multithresh}
 \emph{multithresh} is the crux of the pencil detection technique used
 here. Instead of doing any actual edge detection, we opt to
 separate the image into different intensity
 regions. Experimentally, a value of 2 passed to the function
 performs best. 2 yields threshold levels that when given to
 \emph{imquantize} as is shown below, divides the image into three
 partitions by relative intensity. Further below, I interpret
 those partitions, and colour black only the areas which weren't
 already white. This yields a binary image where only the shapes
 of the pencils are visible, effectively avoiding the need for
 finicky edge detection altogether.\\

With this three-step technique, the algorithm produces images like
the following from a gray-scaled image:

\begin{center}
  \includegraphics[scale=0.4]{two}
  \includegraphics[scale=0.4]{crossed}    
\end{center}

\begin{lstlisting}
  [levels,metric] = multithresh(FI,2);  % Threshold of 2 seems to be best.
  seg_I = imquantize(FI, levels);

  % Sanitize `seg_I` to only show white where pencils were.
  BW = [];
  for i=1:size(seg_I,1)
    for j=1:size(seg_I,2)
      if seg_I(i,j) == 3
        BW(i,j) = 1.0;
      else
        BW(i,j) = 0;
      end
    end
  end
\end{lstlisting}

\subsubsection{Line Discovery - Hough Transform}
The Hough Transform allows us to find long line segments in our
binary image. Before doing that, however, I found it advantageous to first
shrink the white regions found above with the \emph{skel} operation of
\emph{bwmorph}, and then fatten it again just slightly with \emph{dilate}.

\begin{lstlisting}
  BW = bwmorph(BW,'skel',Inf);
  BW = bwmorph(BW,'dilate');
\end{lstlisting}

This tends to produces nice lines that run straight down the center of
the pencil in the final image.\\

\begin{center}
  \includegraphics[scale=0.5]{crossed-thin}  
\end{center}

Next, we perform the actual Hough Transformation:

\begin{lstlisting}
  [H,thetas,ros] = hough(BW);
  peaks = houghpeaks(H,4);
  lines = houghlines(BW,thetas,ros,peaks);
\end{lstlisting}

And then overlay the found lines onto the original image, grayscaled.

\begin{center}
  \includegraphics[scale=0.5]{crossed-lines}
\end{center}

Notice that despite the noticable gap in the lines caused by a shadow in the
previous image, our Hough line bridges that gap and produces a full line.

\subsection{Limitations}
\textbf{Background Colour} This techniques requires a background darker than
the pencils themselves, or the pencils may be lost during the
\emph{multithresh} stage, as can be seen in the picture with six pencils:

\begin{center}
  \includegraphics[scale=0.5]{six-bw}
  \includegraphics[scale=0.5]{six-lines}
\end{center}

Note, this constraint should be possible to satisfy on an assembly line!\\

\textbf{Background Texture} The algorithm produces the following
on the infamous ``carpet'' image:

\begin{center}
  \includegraphics[scale=0.5]{carpet-thresh}
  \includegraphics[scale=0.5]{carpet-lines}
\end{center}

\textbf{Pencil Quantity} There is very likely an upper bound to the number
of pencils it can detect at a time. If we tell it to look for more at once,
the algorithm will often incorrectly find multiple lines per pencil.\\

\textbf{Occlusion} While the algorithm can accurately determine the location
of pencils, it cannot tell which one is above which.

\end{document}
